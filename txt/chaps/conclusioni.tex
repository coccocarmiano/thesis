\chapter{Conclusioni}

Quanto discusso in questa tesi ha permesso di
accertare che delle possibilità di miglioramento
degli attuali strumenti di ambito medico
siano certamente possibili.
Questa possibilità è offerta da miglioramenti
di carattere tecnologico su più fronti,
dall'impiego di modelli di intelligenza
artificiale moderni che seppur imperfetti
e con una carenza di dati da cui apprendere
di gran lunga inferiore a quella che
solitamente ci si aspetterebba si abbia
a disposizione riescono comunque ad ottenre
risultati migliori di modelli che fino
a poco tempo fa erano considerati
i migliori disponibili e che possono,
seppur per costruzione, contare su
molti più dati.
Seppur la ricerca in merito sia tutt'altro
che conclusa, possiamo comunque
constatare che il vero limite che ad 
oggi frena l'impiego di questi modelli
in ambienti reali non sia la capacità
dei modelli di apprendere ad analizzare
le immagini ma bensì le difficoltà
nella raccolta e nell'annotazione
dei dati.
Sebbene come abbiamo visto esistano
dataset pubblici rilasciati allo scopo
di sperimentare e provare e costruire
applicazioni di questo tipo, ogni
condizione medica è diversa dalle altre
e necessita di informazioni coerenti in
merito, seppur sia stato possibile
accertare che le techine di transfer
learning abbiamo comunque una loro
utilità nel calmierare la carenza di 
informazioni.
Nonostante l'intraprendenza che un
singolo ospedale possa avere nel
voler sperimentare nuove tecnologie,
la raccolta e l'annotazione dei dati
è un processo costoso in termini di
tempo che spesso limita i progressi
in tal senso, problema accentuato
in maniera drastica dall'operare in
un settore in cui il livello di
specializzazione richiesto dall'
operatore che deve eseguire tale
annotazione è elevatissimo, costo
reso a sua volta più elevato dalla
bassa incidenza della malattia.

