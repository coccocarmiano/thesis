Il tratto gastro-intestinale umano (GI) è un'organo soggetto a
molteplici svilippi anormali di mucosa o tessuto, che possono
sfociare in malattie di diversa gravità, da lievi disturbi trattabili
fino a forme letali di cancro.
Tra le varie condizioni che è possibile riscontrare, le metaplasie
gastro-intestinali (MGI), una modificazione reversibile di tessuto
che puo' essere sintomo di una condizione pre-tumorale, che
se identificata per tempo è possibile trattare per tempo
prima dello sviluppo di una trasformazione irreversibile.
La {\it World Health Organization} (WHO) stima che ogni anno
3.5 milioni di pazienti incorrano in condizioni tumorali
nel tratto intestinale, che con un tasso di mortalità del
63\% causa circa 2.2 milioni di morti all'anno.

In questo contesto è possibile inserire le intelligenze artificiali.
Il fervente entusiasmo verso questa dottrina e la diffusione
di dataset di riferimento ha permesso a ricercatori e studiosi
di tutto il mondo di realizzare architetture via via più
specializzate nella risoluzione di problemi di vario tipo,
nello specifico quello qui trattato che è l'analisi di immagini,
dove ad oggi esistono decine di adattabili alle esigenze e al tipo
specifico di problema che si vuole risolvere.

Il raggiungimento di risultati oltre le aspettative
da tali modelli ha portato a interrogarsi come poter impiegare
i modelli, fino a poco tempo fa relegati al solo mondo dello studio,
in ambiti reali dove è possibile impiegare questa tecnologia al fine
di efficientare processi ripetitivi o complessi.
L'accuratezza delle reti convoluzionali nel riconoscere {\it pattern}
e la capacità di classificatori probabilistici di discriminare
sulla base di osservazioni statistiche puo' portare a pensare
all'utilità dello sviluppo di uno strumento di supporto,
denominato CAD, {\it computer aided diagnosis}, che non vada
direttamente a sostituirsi alla diagnosi medica ma che aiuti
o renda più efficiente il processo di valutazione.

In questo contesto bisogna anche tener conto delle problematiche
reali che un approccio reale del problema porta con sè.
Un approccio {\it data oriented} ha bisogno di adeguate
quantità di dati, e uno strumento che intende facilitare la vita
di un operatore non puo' richiedere eccessiva conoscenza
aggiuntiva o tempo da investire da parte dell'operatore.

Il lavoro qui proposto è dunque una sintesi dello studio
e degli sforzi compiuti in questa direzione nel tempo recente.
Utilizzando due dataset di riferimento valuteremo le capacità
e le opportunità offerte da due modelli di intelligenza artificiale
in ambito medico, in riferimento proprio al riconoscimento
di metaplasie gastro-intestinali, al contempo valutando
le limitazioni e le esigenze che questi modelli portano con sè.
