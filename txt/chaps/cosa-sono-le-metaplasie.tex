\chapter{Introduzione}
\section{Cosa Sono le Metaplasie}

Prima di entrare nel dettaglio tecnico di quali strumenti
e quali possibilità sono ad oggi offerte nella realizzazione
di un applicativo, è bene avere in mente il problema di riferimento,
studio che va necessariamente condotto a più livelli, come vedremo
di seguito, ma che per iniziare è meglio iniziare dal problema
nel mondo reale.

Il termine metaplasia è un termine spesso usato impropriamente 
se non addirittura erroneamente nel linguaggio comune.
Di fatto, si tende ad accumunare in maniera sistemca
questa condizione medica con la neoplasia, noto più comunemente col termine tumore.
Per quanto entrambe le situazioni siano caratterizzate da anomalie
riguardanti un tipo di tessuto, tipo che dipende ovviamente dall'area di interesse,
le dinamiche e le implicazioni dei due casi sono molto diverse.

Le due condizioni sono però strettamente correlate, dato che una metaplasia
non identificata e curata per tempo puo' per l'appunto portare ad una neoplasia,
dato che la metaplasia è uno stadio intermedio di tessuto differenziato che
è potenzialmente reversibile.
Se questo processo procede fino alla fine, si trasforma in una neoplasia e
non si puo' tornare al tessuto di tipo precedente.
La reversibilità della metaplasia accade nel momento in cui lo stimolo
che ha portato il tessuto a cambiare cessa, ed è resa possibile poiché
la variazione altera la componente genetica, detto fenotipica, mentre il
genoma cellulare viene conservato senza alcuna alterazione, cosa che permette
anche al tessuto di riprendere la sua funzione originale una volta che
il tessuto torna allo stadio originale.

Lo stimolo in questione è solitamente un cambio d'ambiente, che porta le
cellule a cambiare per adattarsi al nuovo.
Uno stimolo comune è dunque l'irritazione dovuto al fumo di sigaretta, che
prevede la trasformazione dell'epitelio cilindrico con uno pluristratificato.
Questo accade in modo che l'organismo risenta meno dalle irritazioi indotte
dal fumo, ma inibisce anche i meccanismi protettivi dell'epitelio stesso che
risulterà quindi più esposto a infezioni e sostanze tossiche provenienti
dall'esterno.

\section{Metaplasie Gastrointestinali}

